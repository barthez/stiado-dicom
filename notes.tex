\documentclass[11pt,a4paper]{article}
\usepackage[utf8]{inputenc}
\usepackage[T1]{fontenc}
\usepackage[polish]{babel}
\usepackage{lmodern}

\begin{document}

\section{Wstęp do formatu DICOM}

\textbf{DICOM} - standard przechowywania, drukowania i przesyłania obrazowych danych
medycznych. Opracowany w roku 1993 jako rozwinięcie standardu ACR/NEMA. Poza
samymi danymi obrazowymi (takimi jak zdjęcia rendgenowskie, USG, zapisami EKG
etc.) Zawiera informacje o pacjencie jak również konstekst i opis danych
obrazowych: opis badania, format i struktura obrazu.

Dane opisowe zapisane są w specjalnie opracowanym \emph{języku}. Format DICOM
ma zdefiniowane 27 standardowych typów danych, określanych jako VR (value
representation). Każdy VR określa długość danych (w bajtach), format zapisu
(binarnie, znakowo) oraz dozwolone znaki. Format DICOM nie ma ustandaryzowanego
kodowania znaków, co może utrudniać między narodową wymianę danych.
Ustandaryzowany jest za to porządek bajtów w zapisie binarnym, domyślnie
Little Endian.

Słownik DICOM definuje stuktury danych okreslające ich przeznaczenie. Struktury
te (nazywane atrybutami, elementami DICOM lub elementami danych DICOM) opisane
są 4 bajtowym tagiem, gdzie dwa pierwsze bajty przyporządkowują go do grupy
danych, a dwa kolejne dokładnie go okreslają. Grupy elementów miały dużo
większe znaczenie w starszych wersjach standardu, teraz, gdy zdefiniowanych
jest ponad 2000 elementów (i ponad to każdy może definiować swoje), grupa
elementu bardzo luźno opisuje jego przynalezność. Każdy element opórcz opisu
danych które reprezentuje, ma też na stałe przypisany VR.

Elementy DICOM mogę być kodowane na dwa sposoby: Implicite i Explicite. Format
Implicite jest krótszy w zapisie ponieważ nie zawiera VR elementów. Pozwala to
zaoszczędzić miejsce/transfer, ale staje się uciązliwe przy wymienianiu danych
które definują własne niestandardowe elementy. Format Explicite słuzży do tego
aby zapisać również VR co pozwala odczytać dane bez potrzeby otwierania
słownika (choć i tak wtedy nie znamy przeznaczenia danych).

\section{S-RGB-8-esopecho.dcm}
Plik US-RGB-8-esopecho.dcm przedstawia jeden obraz z badania Echokardiograficznego.

Plik ten jest kodowany w formacie explicit. Jego wielkość to około 90kb, z
czego danych obrazowych jest 90\%. Plik używa grupowania elementów danych.

Jest to plik powstały podczas badania echokardiograficznego 5 listopada 1994 roku, o godzinie 14:30. Rejestracja została przeprowadzona przy pomocy urządzenia Acme Products.

W tym pliki dane pacjenta zostały usunięte, plik poddano anonimizacji.

Pobieranie danych zostało zrealizowane za pomocą protokołu Quad Capture. Jest to protokół rejestracji dźwięku.

Zawarty w pliku obraz jest obrazem nuer 107 z 2 serii danych.

\textbf{Parametry obrazu.} Każdy piksel jest zapisany za pomocą trzech wartości. Pojedyncza wartość odpowiada jednemu z kolorów: czerwonego, niebieskiego lub zielonego. Piksele ułożone są obok siebie dla danego piksela. Rozmiar obrazu to 120 na 256 pikseli, z których każdy ma kształt prostokąta o proporcjach boków 4 do 3. Zapisane wartości pikseli są bez znaku.

W pliku piskele są zapisane w przedstawiony sposób. Trzy wartości każdego piksela są ułożone obok siebie w pliku.

\section{Wnioski}
W plikach największą cześcią pod względem ilości danych są same obrazy, co oznacza, że dane opisowe o obrazie, dane o pacjencie, badaniu, nie są obciążające dany plik. Dzięki temu przechowywanie obrazów w formacie DICOM nie jest znacząco bardziej wymagające niż przechowywanie samych obrazów. Również przy transmisji obrazów najwięcej danych jest o samym obrazie i nie powoduje to spowolnienia transmisji przez dane opisujące obraz.

Obrazy w większości nie są skompresowane. Jest to spowodowane tym, że przy użyciu kompresji stratnej można by było stracić ważne informacje medyczne. W jednym z trzech plików zastosowano kompresję z pomocą użycia mapy kolorów.

Standard DICOM udostępnia możliwość dokładnego opisu jak powinien wyglądać obraz. Służą do tego specjalne parametry jak rozmiar piksela czy odległość pomiędzy pikselami.

\end{document}
