\documentclass{beamer}
\usepackage[utf8]{inputenc}
\usepackage[T1]{fontenc}
\usepackage[polish]{babel}
\usepackage{graphicx}
\usepackage{lmodern}
\usepackage{url}
\usepackage{hyperref}

\usetheme{AGH}

\title[DICOM]{Format obrazów DICOM na podstawie przykładowych plików}

\author[B. Bułat, T. Drzewiecki]{Bartłomiej Bułat, Tomasz Drzewiecki}

\date[2012]{27.11.2012}

\institute[AGH]
{Wydział EAIiIB\\ 
Katedra Automatyki
}

\setbeamertemplate{itemize item}{$\maltese$}

\begin{document}

{
\usebackgroundtemplate{\includegraphics[width=\paperwidth]{titlepagepl}} % wersja polska
 \begin{frame}
   \titlepage
 \end{frame}
}

%---------------------------------------------------------------------------

\begin{frame}
  \frametitle{Spis treści}
  \tableofcontents
\end{frame}

\newcommand{\nextoc}{ \begin{frame}\frametitle{Spis treści}\tableofcontents[hidesubsections]\end{frame}}


\section{Wstęp}
\begin{frame}
    \frametitle{Wstęp}

\end{frame}

\section{Język DICOM}
\begin{frame}
    \frametitle{Gramatyka DICOM}
\begin{itemize}
  \item Definicja 27 standardowych typów danych
  \item VR (\emph{value representations}) są zdefiniowane w częsci PS3.5 standardu
  \item VR określa dozwolone znaki i długość danych
  \item Każdy VR jest posiada dwuliterowy skrót
\end{itemize}
\end{frame}

%% TODO: Big/Little Endian?

\begin{frame}
    \frametitle{Słownik DICOM}
\begin{itemize}
  \item Definicja ponad 2000 \emph{atrybutów}, (\emph{elementów DICOM, dlementów danych DICOM})
  \item Każdy element ma swój numer (\emph{tag}) opisany jako (Grupa,Element).
  \item Każdy element odpowiada rzeczywistym danym medycznym
  \item Każdy element ma przypisany VR
\end{itemize}
\end{frame}


\begin{frame}
    \frametitle{Kodowanie elementów danych DICOM}


\begin{itemize}
  \item W formacie DICOM istnieją dwa sposoby kodowania elementów danych: \emph{Implicit} i \emph{Explicit}
  \item Format \emph{Implicite} jest bardziej kompaktowy i działa w oparciu o słownik DICOM
  \item Format \emph{Explicit} jest dłuższy, ale nie potrzebuje informacji ze słownika DICOM od poprawnego odczytu danych. Sam określa VR przechowywanych danych.
\end{itemize}
\end{frame}


\begin{frame}
    \frametitle{Format Implicit}
\begin{itemize}
  \item Nazwa grupy, 2 bajtowa liczba całkowita bez znaku
  \item Numer elementy, 2 bajtowa liczba całkowita bez znaku
  \item Długość danych, 4 bajtowa liczba całkowita L
  \item Dane, Parzysta liczba L bajtów zawierająca dane  
\end{itemize}
\end{frame}


\begin{frame}[allowframebreaks]
  \frametitle{Format Explicit}

Dla wszystkich VR oprócz OB, OW, OF, SQ, UT i UN:
\begin{itemize}
  \item Nazwa grupy, 2 bajtowa liczba całkowita bez znaku
  \item Numer elementy, 2 bajtowa liczba całkowita bez znaku
  \item VR, 2 znaki, 2 bajty
  \item Długość danych, 2 bajtowa liczba całkowita L
  \item Dane, Parzysta liczba L bajtów zawierająca dane  
\end{itemize}

\framebreak

Dla OB, OW, OF, SQ, UT i UN:
\begin{itemize}
  \item Nazwa grupy, 2 bajtowa liczba całkowita bez znaku
  \item Numer elementy, 2 bajtowa liczba całkowita bez znaku
  \item VR, 2 znaki, 2 bajty
  \item Zarezerwowane 2 bajty ustwione na wartość \texttt{0x0000}.
  \item Długość danych, 2 bajtowa liczba całkowita L
  \item Dane, Parzysta liczba L bajtów zawierająca dane  
\end{itemize}
\end{frame}

\begin{frame}
    \vskip5em
    \begin{center}
        \Large{Dziękuję za uwagę!}
    \end{center}
\end{frame}

\end{document}

